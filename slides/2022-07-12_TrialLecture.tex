\documentclass[usepdftitle=false, aspectratio=169]{beamer}
\usepackage{pgfpages}

\setbeameroption{hide notes}
\usepackage[theorems]{tcolorbox}

\definecolor{mygreen}{rgb}{.125,.5,.25}
\definecolor{myred}{rgb}{0.8, 0.0, 0.0}
\definecolor{myblue}{rgb}{0,0,.8}

% --- THEME ---
\usetheme[progressbar=frametitle]{metropolis}
\useoutertheme{metropolis}
\useinnertheme{metropolis}
\usefonttheme{metropolis}
\usecolortheme{spruce}  % spruce, metropolis, dove, crane, beaver, seagull
\setbeamercolor{background canvas}{bg=white}
\usecolortheme[named=mygreen]{structure}
\usefonttheme[onlymath]{serif}
\setbeamertemplate{frame numbering}[counter]  % none, counter, fraction

% --- PACKAGES ---
\usepackage[UKenglish]{babel}
\usepackage[utf8]{inputenc}
\usepackage{lmodern}
\usepackage[T1]{fontenc}

% \usepackage{appendixnumberbeamer}
\usepackage{upquote}
\usepackage[straightquotes]{newtxtt}
\usetikzlibrary{positioning}
% \usepackage{minted}
\usepackage{multicol}
\usepackage{xspace}
\usepackage{booktabs}
\usepackage{siunitx}

% --- SETTINGS ---
\graphicspath{{./figures/}}
\setlength{\fboxsep}{0pt}
\frenchspacing
% Avoid font-warning with itemize bullets.
\renewcommand\textbullet{\ensuremath{\bullet}}

% --- OWN COMMANDS ---
\newcommand{\bdra}{\ensuremath{\boldsymbol \Rightarrow }~}
\newcommand{\bdla}{\ensuremath{\boldsymbol \Leftarrow }~}
\newcommand{\dra}{\ensuremath{\Rightarrow }~}
\newcommand{\dla}{\ensuremath{\Leftarrow }~}
\newcommand{\mr}[1]{\mathrm{#1}}
\newcommand{\ohmm}{\ensuremath{\Omega\,}\text{m}\xspace}
\newcommand{\emg}[2]{\texttt{emg#1#2}\xspace}
\newcommand{\emsig}{\texttt{emsig}\xspace}
\newcommand{\mare}{\texttt{MARE2DEM}\xspace}
\newcommand{\geomar}[2]{\texttt{GEOMAR#1#2}\xspace}
\newcommand{\gtem}[2]{\texttt{gTEM#1#2}\xspace}
\newcommand{\empymod}{\texttt{empymod}\xspace}
\newcommand{\simpeg}{\texttt{SimPEG}\xspace}
\newcommand{\pygimli}{\texttt{pyGIMLi}\xspace}
\newcommand{\discretize}{\texttt{discretize}\xspace}
\newcommand{\custem}{\texttt{custEM}\xspace}
\newcommand{\petgem}{\texttt{PETGEM}\xspace}
\newcommand{\rmk}[1]{{\color{red}\bfseries #1}}
\newcommand{\maybe}[1]{{\color{gray} #1}}
\newcommand{\todo}{{\color{myred}\texttt{TODO:}}\xspace}
\newcommand{\bm}[1]{{\mathbf{#1}}}

\newcommand{\code}[1]{\texttt{\color{mygreen}#1}}

% Add page number to slides
\newcommand{\ato}{\addtocounter{framenumber}{1}}

% --- TITLE-STUFF ---
\newcommand{\ttitle}{From scripting to coding in 30 minutes}
\title{\ttitle}
\subtitle{How to refactor, document, and distribute your code (example using
Python)}
\author{Dr. Dieter Werthmüller}
\date{Trial Lecture and Interview; 12 July 2022}
\institute{Dep. Geoscience \& Engineering, CEG, TU Delft}
\hypersetup{
  pdftitle={\ttitle},
  allcolors=mygreen,
 colorlinks=true
}

% --- SLIDES ---
\begin{document}
\metroset{block=fill}  % Fills the block-environment

\ato % ---------------------------------------------------------------------- %
\maketitle

\begin{frame}
  {After today's lecture you should know\ldots}
  \begin{itemize}\itemsep0.5cm
    \item when and why to refactor your scripts/code;
    \item when and why to move from a notebook to a Python file;
    \item how to document your functions;
    \item how to create a module;
    \item how to easily distribute your module.
  \end{itemize}

  ~\\
  \centering
  \alert{\dra Let's start in a Jupyter Notebook!}
\end{frame}

\begin{frame}
  {Checklist}

  \begin{itemize}\itemsep .3cm
    \item If similar statements repeat over and over in a script\\
      \alert{\dra create a function}
    \item \alert{Document your functions}; for your future self and others
    \item If you use similar statements or functions in different notebooks of
      a directory\\
      \alert{\dra create a python-file that you can import}
    \item If you use similar statements or functions in different places\\
      \alert{\dra create a python module that you can import anywhere}
    \item Put it online (e.g. GitHub), so anyone can easily install it via\\
      \code{pip install git+https://github.com/\{username\}/\{reponame\}}
  \end{itemize}
\end{frame}

\begin{frame}
  {Advanced topics and further info}

  \begin{columns}[c]
    \column{.5\textwidth}
      \begin{itemize}
        \item Deploying to PyPi and Conda
        \item Testing (plus linting, coverage)
        \item Continuous Integration
      \end{itemize}

    \column{.5\textwidth}
      \begin{itemize}
        \item Benchmarking
        \item Creating a manual
      \end{itemize}

  \end{columns}

  ~\\[.5cm]
  Further info:
  \begin{itemize}
    \item You can find the material of this lecture on
      \href{https://github.com/prisae/tlecture}{github.com/prisae/tlecture}
    \item For more information about creating a module, see
  \href{https://packaging.python.org/en/latest/tutorials/packaging-projects/}%
       {packaging.python.org/en/latest/tutorials/packaging-projects}
  \end{itemize}

\end{frame}

\end{document}
